\section{Udviklingsværktøjer}
En kort beskrivelse af de anvendte udviklingsværktøjer og de erfaringer der er
gjort med disse.

\begin{enumerate}
	\item TortoiseSVN 1.8.7
			\begin{enumerate}
				\item TortoiseSVN er et program til at versionstyre og dele filer i projektet, SVN kildenten har været koldet op mod au's server. (http://svn.nfit.au.dk/)
			\end{enumerate}
	\item PSoC Creator 3.0 Servies Pack 1
		\begin{enumerate}
			\item PSoC Creator bruges til at bygge software, der programmers til PSoC3.
		\end{enumerate}
	\item Qt
		\begin{enumerate}
			\item Qt er brugt til udviklingen af det grafiske brugerflade på DevKit8000.
		\end{enumerate}
	\item MultiSim 11
		\begin{enumerate}
			\item MultiSim bruges til at simeruler hardware kredsløb samt indtegningen af kredsløb.
		\end{enumerate}
	\item UltiBoard 11
		\begin{enumerate}
			\item UltiBoard er brugt til design af print.
				\end{enumerate}
	\item Rapportskrivning
	\begin{enumerate}
		\item TexStudio (2.7.0 Windows)
			\begin{enumerate}
				\item TexStudio er brugt som skriveprogram til dokumention og rapportskrivning. TexStudio arbejder med filformatten latex(.tex).
			\end{enumerate}
		\item texlive-full (Linux)
			\begin{enumerate}
				\item Bruge til at installere alle "packages" som bliver indporteres i latex.
			\end{enumerate}
		\item basic MikTex 2.9.5105 (Windows)
			\begin{enumerate}
				\item Bruge til at installere alle "packages" som bliver indporteres i latex.
			\end{enumerate}
	\end{enumerate}
	\item Mircosoft Office Visio 2010
		\begin{enumerate}
			\item Visio er bleven brugt til indtegning af diagrammer.
		\end{enumerate}
\end{enumerate}

Størstedelen af projektgruppen har skullet lære at bruge SVN og Latex på et mere anvendeligt niveau end blot kendskab til programmerne, hvilket i starte gav nogle problemer. De resterende programmer er enten blevet lært at kende igennem undervisning eller er kendt fra tidligere semestre. 
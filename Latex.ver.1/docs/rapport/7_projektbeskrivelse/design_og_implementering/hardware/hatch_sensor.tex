\subsection{Lågekontakt - Jens}
\subsubsection{Design}

For at muliggøre registreringen af lågens status er det nødvendig med den form for kontakt, som har følgende egenskaber: Et GPIO i form af 5VDC, samt have den fysiske udførelse der muliggør overvågning af ledningsbrud.

Lågekontaktens signal skal både forbindes til PSoC og DevKit8000, da begge enheder har brug for at kende lågens status.

Dette kan udføres på flere måder, bl.a.
\begin{itemize}
 \item To separate kontakter. En til PSoC3 samt en til DevKit.
 \item En kontakt med et signal til PSoC3, samt et signal til DevKit.
 \item En kontakt med signalet til PSoC3 og herfra videre route signalet til DevKit8000.
\end{itemize}

Løsningen der blev valgt var: \textit{En kontakt med signalet til PSoC3 og herfra videre route signalet til DevKit8000.} 
Dette valg blev fortaget hovedsageligt pga. at det er muligt at fjerne prel fra signalet på PSoC3, således at ekstra hardware komponenter kan udgås.  

\subsubsection{Implementation}

Kontaktprellet bliver fjernet via en debouncer, som er en standard komponent i PSoC3 creator. 

Det er valgt at køre med en relativ lav clock på 10Hz til debouncerne, da det er begrænset hvor hurtigt det er nødvendig at registrere skift i lågens status. 

q udgangen på debounce filteret er ført videre til DevKit8000. Denne udgang er filtreret for kontakt prel. 

Der bliver kaldt en ISR på både den positive og negative flanke, denne ISR toggeler et flag som bliver brugt i "Main" til at overvåge lågens status. 
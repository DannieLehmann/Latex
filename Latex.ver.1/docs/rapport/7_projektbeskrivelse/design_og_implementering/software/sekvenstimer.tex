\subsection{Sekvenstimer - Simon og Stine}

\subsubsection{Design}
Processen at udruge et æg tager op til flere uger, og det er vigtigt at kunne holde styr på de forskellige faser i udrugningen. Derfor er der brug for en timer, som over lang tid kan holde styr på tiden.

Umiddelbart har vi tre muligheder på PSoC3-platformen, som opfylder ovenstående ønsker:

1) En timer som periodisk genererer et interrupt, som kan bruges til at optælle en variabel.

2) En count-down timer som udfører en handling, når timeren har talt ned. Timeren indstilles herefter til tiden indtil den næste handling skal udføres.

3) En Real-Time Clock (RTC), hvor en alarm kan sættes til det næste tidspunkt hvorpå der skal ske en handling.

Mulighed nr. 1 har fordelen, at vi hele tiden har en variabel, som beskriver hvor lang tid der er gået - denne information kan bruges, når brugeren skal informeres om tiden.

\subsubsection{Implementering}
Implementeringen af timeren foregår i PSoC Creator, hvor der vælges en timerkomponent, en clock og en ISR (Interrupt Service Rutine). Clocken sættes til timerens clock-indgang, og ligeledes sættes ISR-komponenten til timerens interrupt-udgang.

Timer og clock indstilles så interruptet fremkommer hvert minut.

I softwaren, i ISR-funktionen, sættes et flag til "true" hver gang timeren genererer et interrupt. Dette flag tjekkes i main-løkken, hvorefter minuttælleren forøges med 1. 
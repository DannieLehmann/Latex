\newpage
\subsection{PSoC main - Simon og Stine}

\subsubsection{Design}

For at forenkle implementeringen af udrugningssekvenser er det blevet besluttet
at starte med kun at implementere forløbet for en enkel ægtype. 
Følgende forløb ønskes:

\begin{itemize}
  \item Tiden nulstilles når udrugningssekvensen påbegyndes.
  \item Temperaturen sættes til 37 grader og luftfugtigheden sættes til 45.
  \item Hver 12. time skal æggene vendes.
  \item Hvert døgn skal det simuleres at hønen forlader æggene i 30 minutter. Dette gøres ved at sætte temperaturen ned til 20 i 30 minutter.
  \item Efter 18 dage skal luftfugtigheden sættes op til 75.
  \item Efter 21 dage er udrugningssekvensen slut. Timeren skal stoppes og DevKit8000 informeres.
\end{itemize}
Forløbet for udrugningssekvensen er inspireret fra hjemmesiden\footnote{Hønsehuset.dk \cite{Rugetips}}.

Forløbet ønskes implementeret som en for-løkke, der bliver kørt igennem en gang i minuttet.
Der skal tages hensyn til de tidsbegænsninger der er i udrugningssekvensen, f.eks. 12 timer og 18 dage.
For at håndtere dette, vælges det at omregne tiden til minutter, hvorefter man vha. if-sætninger, kan håndtere begrænsningerne.

Tidsbegrænsningerne omregnet til minutter:
\begin{itemize}
  \item 12 timer = 720 minutter.
  \item 24 timer = 1440 minutter.
  \item 18 dage = 25920 minutter.
  \item 21 dage = 30240 minutter.
\end{itemize}


\subsubsection{Implementering}
Implementeringen af forløbet for udrugningen blev lavet i et PSoC projekt.

Operatoren modulus blev valgt til at hjælpe med at implementere de tidsbegrænsede dele af forløbet.
En variabel countInMinutes blev oprettet til at indeholde tiden gået siden start.
Da countInMinutes indeholder tiden der er gået, kan modulus bruges til at overholde tidsbegrænsningerne.
Eksempel på brug af modulus: 

\texttt{if(countInMinutes \% 720 == 0)}

Denne if-sætning checker på, om det er tid til æggevending, hvilket er hvert 720. minut.

For hver af punkterne i forløbet (se design afsnit) blev der oprettet en if-sætning, magen til den ovenfor.
Ud over if-sætningerne for 12 timer, 24 timer, 18 dage og 21 dage skulle der oprettes en if-sætning til at håndtere simuleringen 
af at hønen forlader reden. 
Hver 24 time bliver temperaturen sat til en lavere temperatur og den samtidige tid, countInMinutes, bliver gemt i en variabel timeStamp.
Derudover bliver et flag sat.
Dette flag samt timeStamp benyttes i en if-sætning der sørger for at temperaturen kun er sænket i 30 minutter. 

\section{Projektgennemførelse}
%Beskrivelse af hvordan projektet er gennemført med overordnet tidsplan og
%evt. arbejdsfordeling. Hvilken udviklingsmodel eller projektstyringsmetode er
%benyttet? Er der f.eks. anvendt flere iterationer eller sprints kan man her beskrive
%hvorledes disse er defineret.

I starten af projektforløbet blev gruppen enige om en overordnet tidsplan for forløbet. Her skulle medregnes de officielle deadlines for afholdelse af reviews og afleverings dato. Derudover var tidsplanen også med til at give et samlet overblik over hvor mange uger der var til rådighed i projektet. Vha. dette overblik kunne gruppen aftale fælles deadlines for f.eks. færdiggørelse af systemarkitektur og implementering. Dette især for at sikre en passende periode afsat til at afslutte projektet i sidste ende. 
\newline Fra begyndelsen af forløbet blev gruppen ligeledes delt op i mindre grupper, alt efter hvilke ansvarsområder de skulle have. \newline
Det blev besluttet at de tre IKT-studerende skulle stå for størstedelen af software opgaverne, heriblandt programmering af DevKit8000. De tre Elektro- og den ene Stærkstrøm-studerende skulle stå for de hardware mæssige opgaver, som f.eks. varmeregulering.
Derudover skulle en af gruppemedlemmerne agerer Scrum-master og stå for det praktiske i gruppen, såsom aftale mødetidspunkter.

Opdelingen i undergrupperne skulle først ske efter de indledende faser af projektet var gennemført. Her er tale om den indledende fase, hvor projekt-produktet skal findes, specifikationen af krav samt fremstilling af systemarkitekturen. 
Dette skulle gøres i fællesskab, for at skabe fælles overblik og enighed over produktet.


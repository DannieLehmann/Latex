\section{Metoder}
%Beskrivelse af de anvendte arbejdsmetoder og udviklingsprocesser (f.eks.
%hvilken analyse- og designmetode, der er anvendt). Giv læseren et overblik
%over de forskellige metoder, som eksempel SysML og Scrum med relevante
%referencer til yderligere litteratur om emnet. Her er det vigtigt at beskrive
%hvordan I afviger fra teorien i brug af de metoder i har valgt at benytte.

Som udgangspunkt blev arbejdsmetoden Scrum valgt som den overordnede arbejdsmetode i projektet. 
Scrum er en agil arbejdsmetode der ligger vægt på at dele opgaverne op i mindre dele, hvorefter en projektgruppe arbejder på disse dele intensivt i et såkaldt sprint.
I dette projekt blev lagt fokus på at lægge fokus på en enkelt funktionalitet i projektet pr sprint. Det blev anset som meget vigtigt at få en funktionalitet til at virke hele vejen igennem systemet. F.eks skulle der som udgangspunkt oprettes kommunikation imellem DevKit8000, PSoC3 og sensoren. 
Der blev lagt fokus på at holde de ugentlige statusmøder som Scrum-metoden også har fokus på. Disse møder skulle foregå to gange om ugen, hvor hvert gruppemedlem gav en status deres individuelle opgaver.

I begyndelsen af projektet, hvor systemarkitekturen for projektet skulle fastlægges, blev der brugt den såkaldte V-model.
Denne model betegnes som en udvidelse af vandfaldsmodellen. I V-modellen bliver der lagt fokus på at have et overordnet overblik over alle faser i projektforløbet; bl.a design, implementering og test.
V-modellen blev af projektgruppen brugt i den indledende fase af projektet; dvs. under fastlæggelse af projektformulering, kravsspecifikation og systemarkitektur.
Denne metode blev valgt, idet det blev anset som vigtigt at alle gruppemedlemmer havde en overordnet forståelse for det endelige produkt og dens ønskede funktionalitet. 

For at skabe overblik over samtlige dele af systemet, udarbejdes der SysML systemarkitektur. 
SysML diagrammerne danner overblik over hele systemet og hjælper med at danne en nemmere overgang til design og implementeringsfasen. 
Systemarkitekturen er bygget op således, at jo længere frem i arkitekturen man kommer, desto flere detaljer får man for systemet. 
Systemarkitekturen blev som sagt udarbejdet i fællesskab. Dette gjorde at alle ville have en overordnet idé om, hvilke funktionaliteter der skulle være med i det endelige produkt. Derudover fik alle en mulighed for at få indflydelse på designet af produktet. 



\subsection{Jens Rix}
Dette afsnit omhandler hvilket erfaringen undertegnet (Jens Rix) har opnået i gemmen semesterprojektet. På det partiske niveau jeg har arbejdet med i forbindelse med projektet.

Generelt har semesterprojektet været spændende af arbejde med. Temperatur og luftfugtigheds sensor har været akillesene gemmen hele semesterprojeket, det svære blev dette problem aldrig løst ordeneligt. Set i retoperspektiv skulle der havde været udtænkt en plan B fra starten af, til sådande system kritiske komponenter, da problemer med sensoren endte med at forplante sig til rasten af projektet og være hovede årsagen til at rugemaskine ikke blev færdig.

Det er erfaret at det er en klar fordele at bruge komponenter som har en standart kommunikationsprotokol, da dette muliggøre at bruge PSoC'en designeres stander byggeblokke så SPI, I2C osv.

Endvidere er det erfaret at der findes en masse standart integreret hardware komponenter, der muliggøre at holde antallet af fysiske enheder på det absolut minimum.

Der har været et problemer med at samle, de forskellige software dele til semesterprojektet, da hver persons kode ikke kun bestod i en .h og en .c fil som vi ellers havde været vandt til, men også "fysiske komponenter" i PSoC creator topdesign. Det lykkes ikke gruppen at finde en smart måde at løse dette problem.   

Der skulle havde været udføre flere del test. De test der blev fortaget, blev udføret meget rodet, det ville havde været et klar fordel at ført en form for log over disse test, samt at gemmen en separate version er software'et når en test var færdig udført. Således at det altid var muligt at gå tilbage i små skridt når der opstod problemer med software'et. 

    
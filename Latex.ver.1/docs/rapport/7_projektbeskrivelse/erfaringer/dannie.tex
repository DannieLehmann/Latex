

\section{Dannie Lehmann}

Set tilbage over projektets fremgangsmåder, løsninger og valg, Har projektet været meget prættet af manglende fast beskrivelse af hvad produktet skulle bestå af, Hvilket har grundt at der konstant er bleven taget beslutninger og nye valg under udvirkningsfasen. Det vil sige at der har manglet en kunde, en kunde der har et ønske om hvordan det færdige projekt skulle lave, havde nogle faste rammer, således at der undgåes forskellige ønsker om det færdige projekt, hvordan forskellige funktioner skulle udføres. 
Der er bleven brugt godt af tidligere erfaring fra tidligere projekter, særligt igennem arkitektur fasen, dog indtager kunde problemet igen, da det sænker arbejdsprocessen at der skulle være fælles meninger omkran udførelsen projektet. Derfor har vi ikke kunne dele projektet op mindre dele og tildele opgaverne til mindre grupper og derved være mere effektiv. 
Hvad angår projektets færdige udførelse, har flaskehalsen været sensoren, skulle dette projekt eller ligningen gentages skulle man har været mere fokus på flaskehalsen. Yderligere skulle der have været mere fokus på valget at sensoren under undersøgelse af bruge af sensorer, således at man undgår at vælge en sensor som arbejder bedre med de andre dele af projekt. Erfaringen med den første valgte sensor, er tydeliget den kompliceret protokol, som kravede et stort stykke arbejde. at den har været en for stor opgave at skulle få til at virke uden at forstyrrer andre dele af projektets software. Set i backspejlet skulle man have tænkt på at indfører en ydeligere hardware del, en mikroprocessor særligt til styringen af sensoren.
\subsection{Mathias}
Generelt set, ligger mange af de erfaringer der er erhvervet i det at skulle skrive og teste software. Metoden vi har anvendt er hver især at sidde og skrive vores software og så sætte det sammen til sidst og håbe på at det hele virker. vores arbejdsproces er  alt for ukonkret til at få denne form for udvikling til at fungere. Software skal testes bedre, der skal tages højde for interrupt håndtering. Software der bygger pop om interrupts skal igennem en test proces hvor der indgår andet software der bruges interrupts, for at sikre sig at ens software kan håndtere de eventuelle timing issus der kan forekomme når hele systemet sættes sammen til sidst. 

Det blev i ISE beskrevet at når man udvikler software, bruges 50\% af tiden på at teste(1, ved ikke lige hvordan jeg sætter referencer). Når vi skriver software er det nok nærmere 20\% af tiden til design, 60\% til implementering og 20\% til test.

(1) I2ISE kompendie, Software test s.59

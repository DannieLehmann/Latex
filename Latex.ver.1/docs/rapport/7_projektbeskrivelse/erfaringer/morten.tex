\subsubsection{Morten's Erfaringer}

Den nemme løsning er ikke altid så nem, som den giver udtryk for at være, og kan være fyldt med mindst ligeså mange problemer som en løsning der ser ud til at være langt mere besværlig. Det er derfor vigtigt for fremtiden at planlægge kortere udviklingsforløb, og modul og integrationstests skal ligeledes planlægges langt mere fast, så evt. fejl, problemer og mangler kan fanges langt hurtigere i processen. Det er også vigtigt i fremtidige projekter at blive bedre til at træffede begrundede fravalg mht. hvordan et projekt skal stykkes sammen, frem for blot at vælge metoden, som ligger lige for.

Dette gælder også når man beslutter sig for i begyndelsen kun at fastlægge arkitektur og grænseflader overfladisk, for senere hen at iterere over dem - så skal man sørge for at det også sker.

Det er nemt at pege på en enkelt del af dette projekt som værende den største årsag til at det ikke er færdigt: Manglende overholdelse af mantraet "Fail Faster". Der blev brugt for lang tid på udvikling af noget, der i sidste ende ikke kom til at virke, og dermed blev skrottet. Erstatningen for denne kom til at virke relativt hurtigt, men alligevel endte den med at give problemer, som ikke nåede at blive løst grundet manglende tid. Dette resulterede også i manglende test af funktionaliteter, der - med en hvis mængde pessimisme - sikkert heller ikke virker helt som det skal, fordi der er ting man ikke har forudset, men, som det har været tilfældet, ikke kan blive konstateret grundet manglende tid til test.

Men ligeså nemt som det er at pege på dette element som akilleshælen, i lige så høj grad må der stilles spørgsmål til hvorfor der internt i gruppen ikke var røster, der var kritiske (ikke på en negativ måde) overfor de manglende resultater. Og på samme måde kan man stille spørgsmålet, hvordan lærer man at acceptere at et slag er tabt? Hvornår skal man erkende, at det, man arbejder på måske ikke kommer til at virke, og at man nok bør ændre noget?
\chapter{Krav}

\section{Generelt}
Der er i dette projekt oprettet 3 typer af krav. Der er forhåndskrav bestående af krav til brugen af platformtyper samt indhold af sensorer og/eller aktuatorer. Der er Udrugningskrav som består af fundne anbefalinger til et udrugningsforløb. Det endelige valg er taget fra hjemmesiden\footnote{Hønsehuset.dk \cite{Rugetips}}. Den sidste type af krav er personlige krav til systemets funktionalitet, det er her bl.a. kravene til tests af systemet bliver beskrevet. De personlige krav er stærkt fremkomne i de udarbejdede Use Cases, samt i nogle af de ikke funktionelle krav. Der er i rapporten valgt kun at fokusere på Use Casene, og derfor kan de ikke funktionelle krav kun findes i kravspecifikationen i dokumentationsrapporten.

\section{Use Cases}
Kravene for projektet er formuleret ved brug af Use Cases.\newline De anvendte Use Cases er illustreret på figur \ref{fig:usecase_diagram}. Denne figur viser Use Case diagrammet som blev udviklet under projektforløbet.
\begin{figure}[h]
\centering
\includegraphics[width=16cm, scale=0.8, trim= 0mm 0mm 0mm 200mm, clip=true,  angle=0]{6_krav/diagrammer/UseCase_Generel_v1.pdf}
\caption{Use Case diagram}
\label{fig:usecase_diagram}
\end{figure}
\newline Det ses at de funktionelle krav er blevet opdelt i 2 Use Cases, Begynd udrugning og Udrug æg.

\textbf{Begynd udrugning:}
Denne Use Case omhandler setup processen, den er beskrevet ud fra brugerens synspunkt. Den beskriver hvorledes brugeren placerer æg i maskinen, samt den efterfølgende process hvor brugeren vælger æggetypen, og derefter starter udrugningsprocessen svarende til det trufne valg.\newline

\textbf{Udrug æg:}
Denne Use Case følger direkte efter Use Case 1, og omhandler systemets håndtering af udrugningsprocessen, samt den afsluttende fase når den ordinære process er færdiggjort. Der beskrives reguleringen af varme og fugtighed, samt hvordan der tjekkes for tidspunkter for æggevending og udluftning. I den afsluttende fase skal der stadig reguleres temperatur og luftfugtighed, men der skal ikke vendes æg.

\subsection{Aktører}
Der er til Use Casene tre vigtige aktører som interagerer med rugemaskinen:
\begin{itemize}
	\item \textbf{Bruger} \newline
	Dette er brugeren som ønsker at udruge æg. Brugeren er hovedsageligt aktiv i den indledende og afsluttende fase af udrugningen hvor der skal ilægges æg og fjernes kyllinger og/eller ikke udrugede æg.
	\item \textbf{System} \newline
	Systemet er den automatiserede styring af udrugningsprocessen. Det er systemet, der håndterer reguleringen af miljøet og styringen af aktuatorerne der er tilstede i rugemaskinen, samt grænsefladen til brugeren.
	\item \textbf{Teknisk ansvarlig} \newline
	Den tekniske ansvarlige er en person med tilstrækkelig viden om rugemaskinen til at kunne fejlfinde og reparere eventuelle fejl.
	\end{itemize}

%\chapter{Funktionelle krav}

%\section{Use Case diagram}

%\begin{figure}[h]
%\centering
%\includegraphics[width=16cm, scale=1, trim= 0mm 0mm 0mm 200mm, clip=true,  angle=0]{6_krav/diagrammer/UseCase_Generel_v1.pdf}
%\caption{Use Case diagram}
\label{fig:usecase_diagram}
%\end{figure}

%\clearpage
%\section{Aktørbeskrivelse}
%
\subsection{Bruger}

\begin{table}[H]
\centering
\begin{tabular}[\textwidth]{|p{0.20\textwidth}|p{0.80\textwidth}|}
\hline Aktørnavn: & Bruger \\ 
\hline Alternativt navn: & User \\ 
\hline Type: & Primær og sekundær\\ 
\hline Beskrivelse: & 
		\begin{enumerate}
		\item Bruger ønsker at udruge æg
		\end{enumerate} \\ 
\hline
\end{tabular}
\caption{Aktør - Bruger}
\label{tab:usecase-aktoer-bruger}
\end{table}


%
\subsection{System}

\begin{table}[H]
\centering
\begin{tabular}[\textwidth]{|p{0.20\textwidth}|p{0.80\textwidth}|}
\hline Aktørnavn: & System \\ 
\hline Alternativt navn: & Rugemaskine \\ 
\hline Type: & Primær \\ 
\hline Beskrivelse: & 
		\begin{enumerate}
		\item Systemet styrer udrugningsprocesser.
		\item Systemet har til opgave at overvåge temperatur, luftfugtighed samt tidsprogression.
		\end{enumerate} \\ 
\hline
\end{tabular}
\caption{Aktør - System}
\label{tab:usecase-aktoer-system}
\end{table}


%
\subsection{Tekniske ansvarlig}

\begin{table}[H]
\centering
\begin{tabular}[\textwidth]{|p{0.20\textwidth}|p{0.80\textwidth}|}
\hline Aktørnavn: & Tekniske ansvarlig \\ 
\hline Alternativt navn: & Servicetekniker \\ 
\hline Type: & Ekstern \\ 
\hline Beskrivelse: & 
		\begin{enumerate}
		\item Tekniske ansvarlig er en uddannet tekniker, der har en faglig forståelse for opbygning af udrugningsmaskinen.
		\end{enumerate} \\ 
\hline
\end{tabular}
\caption{Aktør - Tekniske ansvarlig}
\label{tab:usecase-aktoer-tekniskeansvarlig}
\end{table}


%\clearpage
%\section{Use Cases}
%\subsection{Begynd udrugning}

\begin{table}[H]
\centering
\begin{tabular}[\textwidth]{|p{0.18\textwidth}|p{0.82\textwidth}|}
\hline Navn & Begynd udrugning \\ 
\hline Usecase ID & \usecaseset{Begynd udrugning} \\ 
\hline Scope &  \\
\hline Primær aktør & Bruger \\ 
\hline Interessenter & Tekniske ansvarlig. \\ 
\hline Forudsætning & Use Case \usecaseref{Udrug aeg} er ikke aktiv \\ 
\hline Resultat & Udrugning af indlagte æg er påbegyndt. \\ 
\hline Hovedforløb &
	\begin{enumerate}
	\item \label{itm:Begynd-step1} Bruger åbner låge.
	\item \label{itm:Begynd-step2} Systemet registrerer åbning af låge. 
	\item \label{itm:Begynd-step3} Systemet låser for interfacet (skriver "låge åben").
	\item \label{itm:Begynd-step4} Bruger placerer æg i maskine.
	\item \label{itm:Begynd-step5} Bruger lukker låge.
	\item \label{itm:Begynd-step6} Systemet registrerer lukning af låge.  
	\item \label{itm:Begynd-step7} Systemet låser op for interfacet (fjerner "låge åben").
	\item \label{itm:Begynd-step8} Bruger vælger type af æg til udrugning (sekvens).
	\item \label{itm:Begynd-step9} Systemet spørger bruger om bekræftelse.
	\item \label{itm:Begynd-step10} Bruger bekræfter valg (vælger "OK").
	\item \label{itm:Begynd-step11} Systemet registrerer valg.
	\item \label{itm:Begynd-step12} Systemet aktiverer sekvensen Udrug Æg (Use Case \usecaseref{Udrug aeg}).
%	\item \label{itm:Begynd-step12} 
	\end{enumerate} \\
	\hline Undtagelser &
	\begin{enumerate}
	\item[\ref{itm:Begynd-step2}a.] System registrerer ikke åbning af låge. 
	\begin{itemize}
	\item Bruger kontakter teknisk ansvarlig. 
	\end{itemize}
	\item[\ref{itm:Begynd-step6}a.] System registrerer ikke lukning af låge.  
	\begin{itemize}
	\item Bruger kontakter teknisk ansvarlig. 
	\end{itemize}
	\item[\ref{itm:Begynd-step10}a.] Bruger bekræfter ikke valg (vælger "Annullér"). 
	\begin{itemize}
	\item Step \ref{itm:Begynd-step8} gentages.
	\end{itemize}
	\end{enumerate} \\
\hline 
\end{tabular}
\caption{Use Case - Begynd udrugning}
\label{tab:usecase-Begynd-udrugning}
\end{table}
%\linespread{1.0}\subsection{Udrug {\ae}g}
\begin{table}[H]
\centering
\begin{tabular}[\textwidth]{|p{0.18\textwidth}|p{0.82\textwidth}|}
\hline Navn & Udrug æg \\ 
\hline Usecase ID & \usecaseset{Udrug aeg} \\ 
\hline Primær aktør & System \\ 
\hline Interessenter & Sekundær aktør: Bruger \\ 
\hline Forudsætning & Use Case \usecaseref{Begynd udrugning} \\ 
\hline Resultat & Æggene er udruget og fjernet fra maskinen \\ 
\hline Hovedforløb &
	\begin{enumerate}
	\item \label{itm:udrugning-step1} Systemet indlæser valgte sekvens.
	\item \label{itm:udrugning-step2} Systemet regulerer temperatur og luftfugtighed.
	\item \label{itm:udrugning-step3} Systemet kontrollerer om det er tid til æggevending.  \newline
	\textbf{hvis} (tid = æggevending): Vend æg.
	\newline
	\textbf{ellers}: Vend ikke æg.
	\newline
	Step \ref{itm:udrugning-step2}-\ref{itm:udrugning-step3} gentages indtil udrugningstiden er afsluttet.
	\item \label{itm:udrugning-step4} 	System informerer brugeren om at udrugningssekvens er færdig.
	\item \label{itm:udrugning-step5} 	Bruger åbner låge.
	\item \label{itm:udrugning-step6} 	Systemet registrerer åbning af låge.
	\item \label{itm:udrugning-step7} 	Systemet afbryder regulering af temperatur og luftfugtighed.
	\item \label{itm:udrugning-step8} 	Bruger fjerner emne(r).
	\item \label{itm:udrugning-step9} 	Bruger lukker låge.
	\item \label{itm:udrugning-step10} 	Systemet registrerer lukning af låge.
	\item \label{itm:udrugning-step11} 	Systemet genoptager regulering af temperatur og luftfugtighed.
	\newline Step \ref{itm:udrugning-step5}-\ref{itm:udrugning-step11} gentages indtil brugeren indikerer overfor systemet at alle emner er fjernet.
	\item \label{itm:udrugning-step12}		Systemet stopper med regulering af temperatur og luftfugtighed.	
	\end{enumerate} \\
\hline Undtagelser &


\begin{enumerate}
		\item[*] Bruger afbryder udrugningen.
		\begin{itemize}
			\item Systemet afbryder Use Case \usecaseref{Udrug aeg}
		\end{itemize}
	\end{enumerate}

	\begin{enumerate}
		\item[*](\ref{itm:udrugning-step2}-\ref{itm:udrugning-step3}) Bruger åbner maskinen.
		\begin{itemize}
			\item Systemet afbryder trin 2-3.
			\begin{itemize}
				\item Bruger lukker maskinen.
				\item Systemet fortsætter med trin 2-3.
			\end{itemize}
		\end{itemize}
	\end{enumerate}

	\begin{enumerate}
	\item[\ref{itm:udrugning-step2}a.] Systemet kan ikke regulere temperatur eller luftfugtighed.
	\begin{itemize}
	\item Systemet registrerer fejl.
	\item Systemet informerer bruger om fejl.
	\end{itemize}
	\end{enumerate} 
	
	
	\begin{enumerate}
	\item[\ref{itm:udrugning-step6}a.]  System registrerer ikke åbning af maskine.
	\begin{itemize}
	\item Bruger kontakter tekniske ansvarlig.
	\end{itemize}
	\end{enumerate}  
	
	\begin{enumerate}
	\item[\ref{itm:udrugning-step10}a.]  System registrerer ikke lukning af maskine.
	\begin{itemize}
	\item Bruger kontakter tekniske ansvarlig.
	\end{itemize}
	\end{enumerate} \\ \hline 
\end{tabular}
\caption{Use Case - Udrug æg}
\label{tab:usecase-Udrug-aeg}
\end{table}
%\clearpage
%\section{Funktionelle krav}
%
\begin{enumerate}
 \item Systemet skal kunne regulere temperaturen efter justering af parameter inden for 10 min. 
 \item Systemet skal kunne regulere luftfugtigheden efter justering af parameter inden for 10 min.
\end{enumerate}
\FloatBarrier

%\section{Ikke-funktionelle krav}
%
\begin{enumerate}

\item Hvis systemet ikke kan overholde de angivne grænser som er beskrevet i de funktionelle krav, skal brugeren informeres vha. alarmer.

\item DevKit8000 8000 anvendes som styrende enhed samt grafisk brugergrænseflade (GUI).

\item PSoC3 og/eller 4 anvendes som grænseflade til føler/sensorer samt aktuatorer.

\item Tilladte afvigelser:
	\begin{itemize}
	\item Temperatur: $\pm$ 1$^\circ$C.
	\item Luftfugtighed : $\pm$ 10 procentpoint.
	\end{itemize}

\item Krav til opstillingsmiljø:
	\begin{itemize}
	\item Temperatur : 15-30$^\circ$C.
	\item Luftfugtighed : <45\% procentpoint.
	\end{itemize}

\item krav til GUI:
	\begin{itemize}
	\item Der skal være mulighed for at navigere via GUI.
	\item GUI skal kunne bruges til at starte og stoppe udrugningen.
	\item Der skal altid på UI under udrugning fremgå: Temperatur, luftfugtighed, tidsprogression.
	\item GUI skal gøre bruger opmærksom på tilstandsændringer.
	\end{itemize}

\item Rugemaskinens dimensioner er: $\pm$ 2 cm
	\begin{itemize}
	\item Brede x cm
	\item Højde x cm
	\item Dybte x cm
	\end{itemize}
		
\item Udrugningsprocedurer skal følge anbefalingerne angivet på www.hønsehus.dk/opdraet/rugetips/

\item Rotation: Rotation på 180$^\circ$ $\pm$ 45$^\circ$

\end{enumerate}

%\clearpage
%\section{Accepttestspecifikation}
%Da vi ikke formåede at sammensætte og teste systemet i dets helhed, kan vi ikke godkende nedenstående accepttests. Vi har dog gennemført og godkendt flere af testene i et "subset" af systemet, som inkluderede sammensatte moduler.

\subsection{Accepttestspecfikation for Use Case \usecaseref{Begynd udrugning}} 

%************************ Use Case 1 **************************************
 
 \subsubsection{Hovedscenarie}
\begin{center}

	\begin{tabular}{| p{3cm} | p{3cm} | p{3cm} | p{3cm} |}
		\hline
		Krav & Udførelse & Forventet resultat & Resultat \\ \hline
		% % % % % % % % % % % % % % % % % % % % % % % % % % % % 
		
		\multirow{2}{3cm}{Ved lågens åbning låses UI. Det forbliver låst indtil lågen lukkes.} 
		& Maskinen står i idle tilstand og lågen åbnes
		& UI låser
		& \\ \cline{2-4}
		
		&Lågen lukkes igen
		
		&UI låser op
		& \\ \hline 
		

	\end{tabular}
\end{center}

%\subsubsection{Undtagelser}
%\begin{center} 
%	\begin{tabular}{| p{3cm} | p{3cm} | p{3cm} | p{3cm} |}
%	\end{tabular}
%\end{center}

\subsection{Accepttestspecfikation for Use Case \usecaseref{Udrug aeg}} 

%************************ Use Case 2 **************************************
 
 \subsubsection{Hovedscenarie}
\begin{center}

	\begin{longtable}{| p{3cm} | p{3cm} | p{3cm} | p{3cm} |}
		\hline
		Krav & Udførelse & Forventet resultat & Resultat \\ \hline
		% % % % % % % % % % % % % % % % % % % % % % % % % % % % 
		
		
		Systemet skal kunne regulere og holde en temperatur indenfor grænserne angivet i ikke funktionelle krav.
		&Systemet opstilles i omgivelser der ligger indenfor de påkrævede rammer. Systemet indstilles til at skulle holde 37$^\circ$C. Systemet sættes i gang og temperaturen aflæses efter 10 minutter. 
		&Maskinen kan regulere og holde temperaturen indenfor de angivne rammer. 
		& \\ \hline
		
		Systemet skal kunne regulere og holde en luftfugtighed indenfor grænserne angivet i ikke funktionelle krav.
		&Systemet opstilles i omgivelser der ligger indenfor de påkrævede rammer. Systemet indstilles til at skulle holde 70\% relativ luftfugtighed. Systemet sættes i gang og luftfugtigheden aflæses efter 10 minutter.
		&Maskinen kan regulere og holde luftfugtigheden indenfor de angivne rammer.
		& \\ \hline
		
		Maskinen kan vende æg med et bestemt interval. Æggene roteres indenfor de i ikke funktionelle krav specificerede grænser.
		&Æg-type specificeres, og systemet indstilles til at skulle vende æggene hvert 5. minut i en periode på 30 minutter. Æggenes vertikale akse markeres med en pil, der peger op. Systemet sættes i gang og kører i 30 minutter. Ved hver vending måles rotation af hvert æg, og deres rotation noteres. Efter hver rotation vendes hvert æg manuelt så orienteringsindikatoren (pilen) peger opad. 
		&Maskinen roterer alle æg indenfor den angivne grænse.
		& \\ \hline		
		
		Systemet skal ved afsluttet udrugning informere brugeren om dette.
		&Maskine indstilles til et 5 minutters udrugningsprogram og igangsættes. Det observeres at systemet informerer brugeren når sekvensen afsluttes.
		&Maskinen vil ved testsekvensens afslutning informere brugeren.
		& \\ \hline

		
		
	\end{longtable}
\end{center}
\clearpage
\subsubsection{Undtagelser}
\begin{center}

	\begin{longtable}{| p{3cm} | p{3cm} | p{3cm} | p{3cm} |}
	\hline
			Krav & Handling & Forventet resultat & Resultat \\ \hline
			% % % % % % % % % % % % % % % % % % % % % % % % % % % % 
			
			\multirow{2}{\linewidth}{Ved lågens åbning låses UI og udrugningssekvensens afbrydes midlertidigt. UI forbliver låst indtil lågen lukkes. Når lågen lukkes genoptages udrugnings- sekvensen.} 
			& Maskinen står I udrug-tilstand og lågen åbnes. \newline
			& UI låser og udrugningssekvensen stoppes.
			& \vspace{2.5cm} \\ \cline{2-4}
			
			&Lågen lukkes igen.
			
			&UI låser op og udrygningssekvensen genoptages.
			& \vspace{2.5cm} \\ \hline 
			
			Hvis det ikke er muligt at overholde temperaturen skal systemet informere brugeren.
			&Maskine opstilles i et lokale med temperatur indenfor de angivne grænser for opstilningsmiljø. Systemet indstilles til at skulle holde temperaturen på 40$^\circ$C, varmelegemet frakobles, og systemet aktiveres. \newline
			&Maskinen informerer brugeren. 
			& \\ \hline
			
			Hvis det ikke er muligt at overholde luftfugtigheden skal systemet informere brugeren.
			
			&Maskine opstilles i et lokale med luftfugtighed indenfor de angivne grænser for opstilningsmiljø. Systemet indstilles til at skulle holde luftfugtigheden på 70\% relativ luftfugtighed, luftfugtighedsreguleringsmekanismen  frakobles, og systemet aktiveres.
			
			&Maskinen informerer brugeren.
			
			& \\ \hline
					
			
		\end{longtable}
	\end{center}
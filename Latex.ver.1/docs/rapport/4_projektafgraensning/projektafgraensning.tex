\chapter{Projektafgrænsning}

%Beskrivelse af projektet set i en større kontekst og de afgrænsninger man har valgt for projektet (prototype, færdiggørelse, udformning osv. osv.). Hvis der er specificeret eller designet mere end implementeret i jeres prototype beskrives det i dette afsnit.


Der er fra projektgruppens side valgt at kører hele projektet digitalt, og derved kunne der undgås at tage stillingen til for meget støj på signalerne. 

Systemet er oprindeligt tænkt til at kunne håndtere flere forskellige typer af æg. Det er valgt at fokusere på kun een type, da dette illustrerer funktionaliteten af systemet, og udvidelse kan udføres på basis af denne prototype.

Følgende ting bliver beskrevet som del af systemet, men er ikke blevet implementeret:

- Den mekaniske del af mekanismen til vending af æg. Mekanismen blev bedømt til ikke at være nødvendig for at demonstrere funktionen til at vende æg, da dette kan illustreres ved at stepmotoren drejer rundt. Hvad angår valget af motor til vendemekanismen, er der ikke taget yderligere forudsætninger hvad ang. moment, omdrejnings hastighed osv. Dette skyldes at det var funktionen ved en stepmotor der skulle fremvises i projektet, frem for det færdige resultat.

- De overordnede, fysiske rammer for rugemaskine, selve kassen. Dette skyldes at der vendemekanismen ikke er bleven designet/bygget, det gør at der ikke kan tages stilling til den de enelige mål. Dog er der anvendt en simple kasse (prototype), således at reguleringen har været mulig og teste. 

- Der er ikke blevet lavet en selvforsynende befugter. Dette skyldes at luftfugtigheden bliver øget ved at forstøve vand ind i rugemaskinen, og skulle der laves et system der selv kan opretholde et fast tryk til forstøvelse, vil opgaven blive for omstænding. Derfor er der fundet en løsning med en trykflaske, med dertil en pumpe som øger trykke i beholderne, som gøre det muligt at forstøve vand ind i kassen.


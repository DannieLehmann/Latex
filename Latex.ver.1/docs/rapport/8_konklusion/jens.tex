\subsection{Jens}
Projektet har været spændende at arbejde med, men til tider også frustrerende, da vores arbejdsproces har været for rodet. Det blev forsøgt at køre Scrum som udviklings metode. Erfaringen med Scrum er at denne form for udviklingsværktøj desværre ikke egner sig til semesterprojekter på IHA pga. de mange side løbene fag og projekter, samt der ikke er resurser til at have en scrum master på projektet.
I fremtidige projekter skal der lægges mere vægt på test. Der skal ligeledes sættes klare deadlines for kritiske elementer således at de ikke ender med at forsinke resten af projektet. Derudover skal der lægges en handlingsplan for hvad man skal gøre hvis disse ikke kan overholdes.

Af de positive ting, vil jeg klart fremhæve beslutningen om at bruge SVN og LaTex til at skrive både rapporten og dokumentation i. Disse udviklingsværktøjer er nogle der bestemt er værd at genbruge til senere projekter.     
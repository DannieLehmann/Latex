\chapter{Konklusion}

\section{Personlige konklusioner}

\subsection{Andreas Laursen}
Der er i dette projekt opnået forståelse for mængden af fejl der kan forekomme på opgaver, som blev anset for simple. Selvom de individuelle funktionaliteter virker tilfredsstillende under en unit test, så kan der opstå mange problemer når de bliver integreret sammen med andre dele af projektet, især når der bruges forskellige protokol kommunikation på samme enhed.

Det er også erfaret at der skal lave flere tests, selv små ændringer i et program kan have gruelige konsekvenser for funktionaliteten, og hvis disse ikke blev fundet med det samme, så kan man bruge lang tid på at finde fejlen i den lille ændring i forhold til den store som var det man testede.

En anden vigtig ting der er erfaret er, at hvis der opstår mangle problemer med en enkelt enhed som kan erstattes af noget andet som stadig overholder vores krav, så gælder det om at have sat en kritisk tidsbegrænsning på at få det til at fungere, så alle funktionaliteter kan opnåes, og gemme forbedringer som udviklingsmuligheder.


\section{Dannie Lehmann}

Set tilbage over projektets fremgangsmåder, løsninger og valg, Har projektet været meget prættet af manglende fast beskrivelse af hvad produktet skulle bestå af, Hvilket har grundt at der konstant er bleven taget beslutninger og nye valg under udvirkningsfasen. Det vil sige at der har manglet en kunde, en kunde der har et ønske om hvordan det færdige projekt skulle lave, havde nogle faste rammer, således at der undgåes forskellige ønsker om det færdige projekt, hvordan forskellige funktioner skulle udføres. 
Der er bleven brugt godt af tidligere erfaring fra tidligere projekter, særligt igennem arkitektur fasen, dog indtager kunde problemet igen, da det sænker arbejdsprocessen at der skulle være fælles meninger omkran udførelsen projektet. Derfor har vi ikke kunne dele projektet op mindre dele og tildele opgaverne til mindre grupper og derved være mere effektiv. 
Hvad angår projektets færdige udførelse, har flaskehalsen været sensoren, skulle dette projekt eller ligningen gentages skulle man har været mere fokus på flaskehalsen. Yderligere skulle der have været mere fokus på valget at sensoren under undersøgelse af bruge af sensorer, således at man undgår at vælge en sensor som arbejder bedre med de andre dele af projekt. Erfaringen med den første valgte sensor, er tydeliget den kompliceret protokol, som kravede et stort stykke arbejde. at den har været en for stor opgave at skulle få til at virke uden at forstyrrer andre dele af projektets software. Set i backspejlet skulle man have tænkt på at indfører en ydeligere hardware del, en mikroprocessor særligt til styringen af sensoren.

\subsection{Jens Rix}
Dette afsnit omhandler hvilket erfaringen undertegnet (Jens Rix) har opnået i gemmen semesterprojektet. På det partiske niveau jeg har arbejdet med i forbindelse med projektet.

Generelt har semesterprojektet været spændende af arbejde med. Temperatur og luftfugtigheds sensor har været akillesene gemmen hele semesterprojeket, det svære blev dette problem aldrig løst ordeneligt. Set i retoperspektiv skulle der havde været udtænkt en plan B fra starten af, til sådande system kritiske komponenter, da problemer med sensoren endte med at forplante sig til rasten af projektet og være hovede årsagen til at rugemaskine ikke blev færdig.

Det er erfaret at det er en klar fordele at bruge komponenter som har en standart kommunikationsprotokol, da dette muliggøre at bruge PSoC'en designeres stander byggeblokke så SPI, I2C osv.

Endvidere er det erfaret at der findes en masse standart integreret hardware komponenter, der muliggøre at holde antallet af fysiske enheder på det absolut minimum.

Der har været et problemer med at samle, de forskellige software dele til semesterprojektet, da hver persons kode ikke kun bestod i en .h og en .c fil som vi ellers havde været vandt til, men også "fysiske komponenter" i PSoC creator topdesign. Det lykkes ikke gruppen at finde en smart måde at løse dette problem.   

Der skulle havde været udføre flere del test. De test der blev fortaget, blev udføret meget rodet, det ville havde været et klar fordel at ført en form for log over disse test, samt at gemmen en separate version er software'et når en test var færdig udført. Således at det altid var muligt at gå tilbage i små skridt når der opstod problemer med software'et. 

    
\subsection{Mathias}
Semesterprojektet dette semester er blevet tilgået med en lidt anden tilgang end tidligere. Denne gang har der været større fokus på at skulle få et funktionelt produkt ud af arbejdet. Dette har også gjort, at fokus på at benytte de forskellige udviklingsværktøjer(UML diagrammer, usecase osv.) ikke har været i fokus. Det har været spændende at skulle stå for hele DevKit8000 siden, både at se hvordan en kommunikation imellem to microcontrollere kan forsimples ved brug at SPI, samt at udvikle et program med et GUI til systemet.

Til at udarbejde programmet til DevKit8000 brugte jeg Qt-creator. Qt er ikke noget vi har beskæftiget os med før, så det har været rigtig interessant at tage et helt nyt udviklingsværktøj i brug, med store ændringer i C++ sproget, og selv skulle tillære sig det. Megen af interessen i forhold til Qt kommer specielt i det, at man kan se flere af de ting vi har lært i I3ISU blive anvendt i Qt frameworket. Dette er fx flertrådet programmering i C++, message distribution systemer og tråd synkronisering(mutex). 

Endnu en gang forsøgte vi at anvende Scrum, men jeg synes personligt det fejlede. Specielt er det manglen på en person som kun agere Scrum master.

Alt i alt, har det for mig været et læreridt projekt, og det har været både frustrerende, men også læreridt ikke at kunne få produktet til at virke i den sidste testfase. Bl.a. er det tydeligt, at vi generelt tester vores software for dårligt. Men den viden må man så taget med videre til næste semester.


\subsubsection{Morten's Erfaringer}

Den nemme løsning er ikke altid så nem, som den giver udtryk for at være, og kan være fyldt med mindst ligeså mange problemer som en løsning der ser ud til at være langt mere besværlig. Det er derfor vigtigt for fremtiden at planlægge kortere udviklingsforløb, og modul og integrationstests skal ligeledes planlægges langt mere fast, så evt. fejl, problemer og mangler kan fanges langt hurtigere i processen. Det er også vigtigt i fremtidige projekter at blive bedre til at træffede begrundede fravalg mht. hvordan et projekt skal stykkes sammen, frem for blot at vælge metoden, som ligger lige for.

Dette gælder også når man beslutter sig for i begyndelsen kun at fastlægge arkitektur og grænseflader overfladisk, for senere hen at iterere over dem - så skal man sørge for at det også sker.

Det er nemt at pege på en enkelt del af dette projekt som værende den største årsag til at det ikke er færdigt: Manglende overholdelse af mantraet "Fail Faster". Der blev brugt for lang tid på udvikling af noget, der i sidste ende ikke kom til at virke, og dermed blev skrottet. Erstatningen for denne kom til at virke relativt hurtigt, men alligevel endte den med at give problemer, som ikke nåede at blive løst grundet manglende tid. Dette resulterede også i manglende test af funktionaliteter, der - med en hvis mængde pessimisme - sikkert heller ikke virker helt som det skal, fordi der er ting man ikke har forudset, men, som det har været tilfældet, ikke kan blive konstateret grundet manglende tid til test.

Men ligeså nemt som det er at pege på dette element som akilleshælen, i lige så høj grad må der stilles spørgsmål til hvorfor der internt i gruppen ikke var røster, der var kritiske (ikke på en negativ måde) overfor de manglende resultater. Og på samme måde kan man stille spørgsmålet, hvordan lærer man at acceptere at et slag er tabt? Hvornår skal man erkende, at det, man arbejder på måske ikke kommer til at virke, og at man nok bør ændre noget?
\subsection{Simon}

Rugemaskinens sekvensrutine blev successfuldt implementeret og testet for hønseæg, således at de tidsbestemte handlinger for denne type æg udføres rettidig. F.eks. at rotere æggene hver tolvte time, og udlufte æggene en gang i døgnet.

Der blev udfærdiget en datakommunikations-protokol til SPI-kommunikationen mellem DevKit8000 og PSoC3. Denne blev implementeret og testet, således at brugeren via touch-interface på DevKit8000 kan styre samt kontrollere udrugningsprogrammet på PSoC3.

Det lykkedes ikke at gøre sekvensrutinen generisk, således at den kan håndtere udruninger af andre ægtyper end hønseæg. Det lykkedes heller ikke at færdiggøre implementeringen af fejlhåndtering, således at brugeren kan se fejlbeskeder. Begge dele pga. mangel på tid.
\clearpage
\subsection{Stine}

3. semesterprojektet har for mig personligt været en god oplevelse. Jeg synes at vores gruppe har fungeret fint sammen, og at de fleste har levet op til de krav, der er stillet.
Da de fleste i gruppen også arbejdede sammen på 2. semesterprojektet, var det nemt at arbejde sammen og afprøve nye ting, som vi syntes skulle forbedres sidste semester.
En overordnet mangel i vores projektarbejde var, at vi ikke var gode nok til at teste de enkelte undergruppers funktionaliteter sammen. Dette gjorde at den samlede test til sidst i forløbet blev en lang og udtrukken proces.
Det er helt klart noget vi må tage til efterretning på de næste projekter vi skal lave. 

Igen forsøgte vi at bruge Scrum som arbejdsmetode. På dette semester gik vi mere målrettet efter at holde de ugentlige statusmøder og at vi fik en bedre rød tråd igennem implementeringsfasen.
For mig fungerede de ugentlige statusmøder rigtig godt, og jeg synes at gruppen var god til at give input og sparre med hinanden.
Den røde tråd igennem forløbet synes jeg forsvandt lidt. I det hele taget blev Scrum ikke fulgt, udover statusmøderne. 
Tilgengæld synes jeg at vi skal blive bedre til at sørge for at deadlines bliver overholdt og bedre til at skride til handling, hvis man kan se at en deadline ikke kan nås.

\section{Generel konklusion}
Gruppens generelle konklusion bygger især på manglende brug af test i implementerings fasen.
Der skal afsættes mere konkret tid af til at teste, og disse tests skal udførers i mindre dele, istedet for at teste hele produktet samlet.

Derudover skal der i fremtiden være mere fokus på at lægge klare deadlines for kritiske elementer i projektet. Der skal tages hurtigere beslutninger angående valg og fravalg af elementer, således at man, hvis man kommer i problemer, ikke sidder fast for længe, og dermed spilder en masse tid.

Det blev valgt at benytte Scrum arbejdsmetoden, men denne blev ikke fulgt. En af grundene til dette anses for at være de andre fag på semestret. For at Scrum kan fungerer optimalt skal der være en Scrum Master, hvis eneste rolle er at styre slagets gang. Da alle i gruppen forventes at have taget aktiv del i forløbet, er det svært at have en enkel person, som kun agerer Scrum Master. Derudover er der de fem fag på semestret, der også fylder. Dette gør, at der ikke er den samme mulighed for at udfører sprints og daglige møder, som man kunne håbe. 



\subsection{Andreas}
I projektet har jeg haft succesoplevelser men også mange frustrerende tidspunkter, hvor der er oplevet fejl som ikke gav logisk mening. Dette kunne være overseelser, som gav frustrationer nok til at revidere éns program til tidligere versioner, og derved føle at ens tid havde været forgæves.

Projektet startede godt ud med fælles diskussioner. Disse opstod især fordi der ingen erfaring var indenfor emnet med udrugning af høns. Da projektet nåede udviklingsfasen opstod der mange problemer i form af at flere overvejelser blev bortkastet pga. ar der var nem adgang til andre løsninger. Dette gjorde, at der blev mistet overblik over produktets sammenhæng som kan have haft en indvirkning på vores integrationsfase af produktet.

Ting jeg vil tage med videre til senere semestre er, at prøve at holde sig til de oprindelige idéer som blev opsat i den lange udviklingsfase. Der skal også sættes mere kritiske tidsbegrænsninger på opgaver så fejlfyldte dele af projektet kan løses tidligt, eller kasseres i tide til at rette op på den spildte tid. Der  skal også laves flere tests, især mindre integrationstests tidligt i forløbet vil kunne sikre en bedre afsluttende fase på projektet.
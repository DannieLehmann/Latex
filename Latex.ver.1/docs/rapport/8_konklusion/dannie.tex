\subsection{Dannie}

Semesterprojektet har denne gang haft nye udfordringer, særligt som formand for en ny gruppe, da jeg ikke tidligere har arbejdet med størstedelen af gruppen. Det giver nogle problemer med at fordele arbejde ud fra folks stærke sider. 
Semesterprojektet på 3. semester er et frit valg, hvilket har givet muligheden for at lave et spændende produkt, men da projektet ikke har været godt nok defineret fra starten har der været mange tolkninger af forskellige opgaver, hvilket også giver mange forskellige løsningsforslag. Det har givet mange gode diskutioner under vejs, og gjort at mange af beslutningerne skulle tages fælles. Projektet havde været forløbet endnu hurtigere hvis definitionerne havde lagt fast. Yderligere vil man nemmere kunne dele projektet op i mindre dele. 
Under implementeringen  har jeg stået for en sensor til at registrering af luftfugtighed og temperatur. Den første valgte sensor havde dog en besværlig protokol at lave program til, og set i bakspejlet skulle den have været håndteret anderledes. PSoC3 har ikke været muligt at lave program til sensoren da den kræver meget præcise signaler, dette kunne have været løst ved at mikroprocessorer til at håndtere sensoren selv. Det var en tidskravende læringsproces, og kostede mange timer der kunne være brugt bedre. 
Den anden valgte sensor havde en standard I2C protokol og var derfor forholdsvist nem at håndtere. 

Overordet set har projektet ikke været en succes for mit vedkommende. Jeg havde svært ved at holde rollen som formand (og Scrum master) særligt fordi denne process kræver ekstra meget arbejde, når folk deler sig op, for at løse forskellige opgaver, samtidigt med at jeg rodede med sensor, som jeg ikke kunne få til at virker. Dog har processen været meget lærerig. 
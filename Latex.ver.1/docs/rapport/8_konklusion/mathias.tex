\subsection{Mathias}
Semesterprojektet dette semester er blevet tilgået med en lidt anden tilgang end tidligere. Denne gang har der været større fokus på at skulle få et funktionelt produkt ud af arbejdet. Dette har også gjort, at fokus på at benytte de forskellige udviklingsværktøjer(UML diagrammer, usecase osv.) ikke har været i fokus. Det har været spændende at skulle stå for hele DevKit8000 siden, både at se hvordan en kommunikation imellem to microcontrollere kan forsimples ved brug at SPI, samt at udvikle et program med et GUI til systemet.

Til at udarbejde programmet til DevKit8000 brugte jeg Qt-creator. Qt er ikke noget vi har beskæftiget os med før, så det har været rigtig interessant at tage et helt nyt udviklingsværktøj i brug, med store ændringer i C++ sproget, og selv skulle tillære sig det. Megen af interessen i forhold til Qt kommer specielt i det, at man kan se flere af de ting vi har lært i I3ISU blive anvendt i Qt frameworket. Dette er fx flertrådet programmering i C++, message distribution systemer og tråd synkronisering(mutex). 

Endnu en gang forsøgte vi at anvende Scrum, men jeg synes personligt det fejlede. Specielt er det manglen på en person som kun agere Scrum master.

Alt i alt, har det for mig været et læreridt projekt, og det har været både frustrerende, men også læreridt ikke at kunne få produktet til at virke i den sidste testfase. Bl.a. er det tydeligt, at vi generelt tester vores software for dårligt. Men den viden må man så taget med videre til næste semester.


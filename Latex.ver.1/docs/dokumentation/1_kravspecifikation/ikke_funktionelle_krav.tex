
\begin{enumerate}

\item Hvis systemet ikke kan overholde de angivne grænser som er beskrevet i de funktionelle krav, skal brugeren informeres vha. alarmer.

\item DevKit8000 8000 anvendes som styrende enhed samt grafisk brugergrænseflade (GUI).

\item PSoC3 og/eller 4 anvendes som grænseflade til føler/sensorer samt aktuatorer.

\item Tilladte afvigelser:
	\begin{itemize}
	\item Temperatur: $\pm$ 1$^\circ$C.
	\item Luftfugtighed : $\pm$ 10 procentpoint.
	\end{itemize}

\item Krav til opstillingsmiljø:
	\begin{itemize}
	\item Temperatur : 15-30$^\circ$C.
	\item Luftfugtighed : <45\% procentpoint.
	\end{itemize}

\item krav til GUI:
	\begin{itemize}
	\item Der skal være mulighed for at navigere via GUI.
	\item GUI skal kunne bruges til at starte og stoppe udrugningen.
	\item Der skal altid på UI under udrugning fremgå: Temperatur, luftfugtighed, tidsprogression.
	\item GUI skal gøre bruger opmærksom på tilstandsændringer.
	\end{itemize}

\item Rugemaskinens dimensioner er: $\pm$ 2 cm
	\begin{itemize}
	\item Brede x cm
	\item Højde x cm
	\item Dybte x cm
	\end{itemize}
		
\item Udrugningsprocedurer skal følge anbefalingerne angivet på www.hønsehus.dk/opdraet/rugetips/

\item Rotation: Rotation på 180$^\circ$ $\pm$ 45$^\circ$

\end{enumerate}
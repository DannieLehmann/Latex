%Usecase and accepttest numbering
%% Se http://en.wikibooks.org/wiki/LaTeX/Counters http://en.wikibooks.org/wiki/LaTeX/Labels_and_Cross-referencing og http://en.wikibooks.org/wiki/LaTeX/Macros for info på hvordan nedenståen er opstået

\newcounter{usecases} %counter
\newcommand{\usecaseset}[1]{\#\refstepcounter{usecases}\arabic{usecases}\label{usecase:#1}} %Tacking an usecase
\newcommand{\usecaseref}[1]{\ref{usecase:#1}} %referencing a usecase

\newcounter{accepttests} %counter
\newcommand{\accepttestset}[1]{\#\refstepcounter{accepttests}\arabic{accepttests}\label{accepttest:#1}}
\newcommand{\accepttestref}[1]{\ref{accepttest:#1}}

%Inkludering af class diagram
\newcommand{\includeclassdiagram}[2]{\includegraphics[width=#1,clip=true,trim=39
40 39 40]{#2}}

%Ens navne på x10 enheder
\newcommand{\xti}{X10}
\newcommand{\xtic}{X10-controller}
\newcommand{\xtie}{X10-enhed}
\newcommand{\xtis}{X10-slave}

\newsubfloat{figure}

%%%% CUSTOM SETTINGS %%%%

\pdfoptionpdfminorversion=6					% Muliggoer inkludering af pdf dokumenter, af version 1.6 og hoejere
\pretolerance=2500 							% Justering af afstand mellem ord (hoejt tal, mindre orddeling og mere luft mellem ord)

% ¤¤ Marginer ¤¤ %
\setlrmarginsandblock{3.5cm}{2.5cm}{*}		% \setlrmarginsandblock{Indbinding}{Kant}{Ratio}
\setulmarginsandblock{2.5cm}{3.0cm}{*}		% \setulmarginsandblock{Top}{Bund}{Ratio}
\checkandfixthelayout 						% Oversaetter vaerdier til brug for andre pakker

%	¤¤ Afsnitsformatering ¤¤ %
\setlength{\parindent}{0mm}           		% Stoerrelse af indryk
\setlength{\parskip}{3mm}          			% Afstand mellem afsnit ved brug af double Enter
\linespread{1,1}							% Linie afstand

% ¤¤ Litteraturlisten ¤¤ %
%\bibpunct[,]{[}{]}{;}{a}{,}{,} 				% Definerer de 6 parametre ved Harvard
% henvisning (bl.a. parantestype og seperatortegn) \bibliographystyle{bibtex/harvard}			% Udseende af litteraturlisten.
\bibliographystyle{plain}			% Udseende af litteraturlisten.

% ¤¤ Indholdsfortegnelse ¤¤ %
\setsecnumdepth{subsection}		 			% Dybden af nummerede overkrifter (part/chapter/section/subsection)
\maxsecnumdepth{subsection}					% Dokumentklassens graense for nummereringsdybde
\settocdepth{subsection} 					% Dybden af indholdsfortegnelsen

% ¤¤ Lister ¤¤ %
\setlist{
  topsep=0pt,								% Vertikal afstand mellem tekst og listen
  itemsep=-1ex,								% Vertikal afstand mellem items
} 

% ¤¤ Visuelle referencer ¤¤ %
\usepackage[colorlinks]{hyperref}			% Danner klikbare referencer (hyperlinks) i dokumentet.
\hypersetup{colorlinks = true,				% Opsaetning af farvede hyperlinks (interne links, citeringer og URL)
    linkcolor = black,
    citecolor = black,
    urlcolor = black
}

% % Sprog opsætning for referencer
\def\figureautorefname{Figur}
\def\tableautorefname{Tabel}

% ¤¤ Opsaetning af figur- og tabeltekst ¤¤ %
\captionnamefont{\small\bfseries\itshape}	% Opsaetning af tekstdelen ('Figur' eller 'Tabel')
\captiontitlefont{\small}					% Opsaetning af nummerering
\captiondelim{. }							% Seperator mellem nummerering og figurtekst
\hangcaption								% Venstrejusterer flere-liniers figurtekst under hinanden
\captionwidth{\linewidth}					% Bredden af figurteksten
\setlength{\belowcaptionskip}{10pt}			% Afstand under figurteksten
		
% ¤¤ Navngivning ¤¤ %
\addto\captionsdanish{
	\renewcommand\appendixname{Appendiks}
	\renewcommand\contentsname{Indholdsfortegnelse}	
	\renewcommand\appendixpagename{Appendiks}
	\renewcommand\appendixtocname{Appendiks}
	\renewcommand\cftchaptername{\chaptername~}				% Skriver "Kapitel" foran kapitlerne i indholdsfortegnelsen
	\renewcommand\cftappendixname{\appendixname~}			% Skriver "Appendiks" foran appendiks i indholdsfortegnelsen
}

% ¤¤ Kapiteludssende ¤¤ %
\definecolor{numbercolor}{gray}{0.7}		% Definerer en farve til brug til kapiteludseende
\newif\ifchapternonum

\makechapterstyle{jenor}{					% Definerer kapiteludseende frem til ...
  \renewcommand\beforechapskip{0pt}
  \renewcommand\printchaptername{}
  \renewcommand\printchapternum{}
  \renewcommand\printchapternonum{\chapternonumtrue}
  \renewcommand\chaptitlefont{\fontfamily{pbk}\fontseries{db}\fontshape{n}\fontsize{25}{35}\selectfont\raggedleft}
  \renewcommand\chapnumfont{\fontfamily{pbk}\fontseries{m}\fontshape{n}\fontsize{1in}{0in}\selectfont\color{numbercolor}}
  \renewcommand\printchaptertitle[1]{%
    \noindent
    \ifchapternonum
    \begin{tabularx}{\textwidth}{X}
    {\let\\\newline\chaptitlefont ##1\par} 
    \end{tabularx}
    \par\vskip-2.5mm\hrule
    \else
    \begin{tabularx}{\textwidth}{Xl}
    {\parbox[b]{\linewidth}{\chaptitlefont ##1}} & \raisebox{-15pt}{\chapnumfont \thechapter}
    \end{tabularx}
    \par\vskip2mm\hrule
    \fi
  }
}											% ... her

\chapterstyle{jenor}						% Valg af kapiteludseende - Google 'memoir chapter styles' for alternativer

% ¤¤ Sidehoved ¤¤ %

\makepagestyle{AAU}							% Definerer sidehoved og sidefod udseende frem til ...
\makepsmarks{AAU}{%
	\createmark{chapter}{left}{shownumber}{}{. \ }
	\createmark{section}{right}{shownumber}{}{. \ }
	\createplainmark{toc}{both}{\contentsname}
	\createplainmark{lof}{both}{\listfigurename}
	\createplainmark{lot}{both}{\listtablename}
	\createplainmark{bib}{both}{\bibname}
	\createplainmark{index}{both}{\indexname}
	\createplainmark{glossary}{both}{\glossaryname}
}
\nouppercaseheads											% Ingen Caps oenskes

\makeevenhead{AAU}{Aarhus Universitet - Gruppe 08}{}{\leftmark}				% Definerer lige siders sidehoved (\makeevenhead{Navn}{Venstre}{Center}{Hoejre})
\makeoddhead{AAU}{\rightmark}{}{Gruppe 08 - Aarhus Universitet}		% Definerer ulige siders sidehoved (\makeoddhead{Navn}{Venstre}{Center}{Hoejre})
\makeevenfoot{AAU}{\thepage}{}{}							% Definerer lige siders sidefod (\makeevenfoot{Navn}{Venstre}{Center}{Hoejre})
\makeoddfoot{AAU}{}{}{\thepage}								% Definerer ulige siders sidefod (\makeoddfoot{Navn}{Venstre}{Center}{Hoejre})
\makeheadrule{AAU}{\textwidth}{0.5pt}						% Tilfoejer en streg under sidehovedets indhold
\makefootrule{AAU}{\textwidth}{0.5pt}{1mm}					% Tilfoejer en streg under sidefodens indhold

\copypagestyle{AAUchap}{AAU}								% Sidehoved for kapitelsider defineres som standardsider, men med blank sidehoved
\makeoddhead{AAUchap}{}{}{}
\makeevenhead{AAUchap}{}{}{}
\makeheadrule{AAUchap}{\textwidth}{0pt}
\aliaspagestyle{chapter}{AAUchap}							% Den ny style vaelges til at gaelde for chapters
															% ... her
															
\pagestyle{AAU}												% Valg af sidehoved og sidefod


%%%% CUSTOM COMMANDS %%%%

% ¤¤ Billede hack ¤¤ %
\newcommand{\figur}[4]{
		\begin{figure}[H] \centering
			\includegraphics[width=#1\textwidth]{billeder/#2}
			\caption{#3}\label{#4}
		\end{figure} 
}

% ¤¤ Specielle tegn ¤¤ %
\newcommand{\grader}{^{\circ}\text{C}}
\newcommand{\gr}{^{\circ}}
\newcommand{\g}{\cdot}


%%%% ORDDELING %%%%

\hyphenation{}

% % listings setup
\lstset{
tabsize=4,
frame=single,
columns=fullflexible,
linewidth=\textwidth
}

% % % sørg for at floats er på siden inden en ny section starter
%\newcommand{\sectionbreak}{\clearpage}
%\newcommand{\subsectionbreak}{\FloatBarrier}

%Code handling - Colors lstlisting

\lstset{language=C++,
                basicstyle=\ttfamily,
                keywordstyle=\color{blue}\ttfamily,
                stringstyle=\color{red}\ttfamily,
                commentstyle=\color{green}\ttfamily,
                morecomment=[l][\color{magenta}]{\#}
}

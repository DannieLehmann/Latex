% ¤¤ Oversaettelse og tegnsaetning ¤¤ %
\usepackage[utf8]{inputenc}					% Input-indkodning af tegnsaet (UTF8)
\usepackage[danish]{babel}					% Dokumentets sprog
\usepackage[T1]{fontenc}					% Output-indkodning af tegnsaet (T1)
\usepackage{ragged2e,anyfontsize}			% Justering af elementer
%\usepackage{fixltx2e}						% Retter forskellige fejl i LaTeX-kernen

\usepackage{lmodern}

\usepackage{datetime} % Nuværende tidspunkt

% ¤¤ Figurer og tabeller (floats) ¤¤ %
\usepackage{graphicx} 						% Haandtering af eksterne billeder (JPG, PNG, EPS, PDF)s
%\usepackage{eso-pic}						% Tilfoej billedekommandoer paa hver side
\usepackage{wrapfig}						% Indsaettelse af figurer omsvoebt af tekst. \begin{wrapfigure}{Placering}{Stoerrelse}
%\usepackage{subcaption}					% Included i memoir class

\usepackage{multirow}                		% Fletning af raekker og kolonner (\multicolumn og \multirow)
\usepackage{multicol}         	        	% Muliggoer output i spalter
\usepackage{rotating}						% Rotation af tekst med \begin{sideways}...\end{sideways}
\usepackage{colortbl} 						% Farver i tabeller (fx \columncolor og \rowcolor)
\usepackage{xcolor}							% Definer farver med \definecolor. Se mere: http://en.wikibooks.org/wiki/LaTeX/Colors
\usepackage{flafter}						% Soerger for at floats ikke optraeder i teksten foer deres reference
\let\newfloat\relax 						% Justering mellem float-pakken og memoir
\usepackage{float}							% Muliggoer eksakt placering af floats, f.eks. \begin{figure}[H]
\usepackage{longtable} %Long tables

\usepackage{afterpage} %Bruges til at få billeder og figurer til at være samen med kapitlet de hører til
\usepackage{placeins}

% ¤¤ Matematik mm. ¤¤
\usepackage{amsmath,amssymb,stmaryrd} 		% Avancerede matematik-udvidelser
\usepackage{mathtools}						% Andre matematik- og tegnudvidelser
\usepackage{textcomp}                 		% Symbol-udvidelser (f.eks. promille-tegn med \textperthousand )
\usepackage{rsphrase}						% Kemi-pakke til RS-saetninger, f.eks. \rsphrase{R1}
\usepackage[version=3]{mhchem} 				% Kemi-pakke til flot og let notation af formler, f.eks. \ce{Fe2O3}
\usepackage{siunitx}						% Flot og konsistent praesentation af tal og enheder med \si{enhed} og \SI{tal}{enhed}
\sisetup{locale=DE}							% Opsaetning af \SI (DE for komma som decimalseparator) 

% ¤¤ Referencer og kilder ¤¤ %
\usepackage[danish]{varioref}				% Muliggoer bl.a. krydshenvisninger med sidetal (\vref)
%\usepackage{natbib}							% Udvidelse med naturvidenskabelige citationsmodeller
%\usepackage{xr}							% Referencer til eksternt dokument med \externaldocument{<NAVN>}
%\usepackage{glossaries}					% Terminologi- eller symbolliste (se mere i Daleifs Latex-bog)

% ¤¤ Misc. ¤¤ %
\usepackage{lipsum}							% Dummy text \lipsum[..]
\usepackage[shortlabels]{enumitem}			% Muliggoer enkelt konfiguration af lister
\usepackage{pdfpages}						% Goer det muligt at inkludere pdf-dokumenter med kommandoen \includepdf[pages={x-y}]{fil.pdf}

%EMF to PDF conversion
\usepackage{epstopdf}

% Kommentarer og rettelser med \fxnote. Med 'final' i stedet for 'draft' udloeser hver note en error i den faerdige rapport.
\usepackage[footnote,danish,final,nomargin]{fixme}

% Microtype gør at teksten ser pænere ud.
\usepackage{microtype}

\usepackage{listings}
\usepackage{titlesec}
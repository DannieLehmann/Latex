 \documentclass[a4paper,11pt,fleqn,dvipsnames,oneside,openright]{memoir} 	%
% ¤¤ Oversaettelse og tegnsaetning ¤¤ %
\usepackage[utf8]{inputenc}					% Input-indkodning af tegnsaet (UTF8)
\usepackage[danish]{babel}					% Dokumentets sprog
\usepackage[T1]{fontenc}					% Output-indkodning af tegnsaet (T1)
\usepackage{ragged2e,anyfontsize}			% Justering af elementer
%\usepackage{fixltx2e}						% Retter forskellige fejl i LaTeX-kernen

\usepackage{lmodern}

\usepackage{datetime} % Nuværende tidspunkt

% ¤¤ Figurer og tabeller (floats) ¤¤ %
\usepackage{graphicx} 						% Haandtering af eksterne billeder (JPG, PNG, EPS, PDF)s
%\usepackage{eso-pic}						% Tilfoej billedekommandoer paa hver side
\usepackage{wrapfig}						% Indsaettelse af figurer omsvoebt af tekst. \begin{wrapfigure}{Placering}{Stoerrelse}
%\usepackage{subcaption}					% Included i memoir class

\usepackage{multirow}                		% Fletning af raekker og kolonner (\multicolumn og \multirow)
\usepackage{multicol}         	        	% Muliggoer output i spalter
\usepackage{rotating}						% Rotation af tekst med \begin{sideways}...\end{sideways}
\usepackage{colortbl} 						% Farver i tabeller (fx \columncolor og \rowcolor)
\usepackage{xcolor}							% Definer farver med \definecolor. Se mere: http://en.wikibooks.org/wiki/LaTeX/Colors
\usepackage{flafter}						% Soerger for at floats ikke optraeder i teksten foer deres reference
\let\newfloat\relax 						% Justering mellem float-pakken og memoir
\usepackage{float}							% Muliggoer eksakt placering af floats, f.eks. \begin{figure}[H]
\usepackage{longtable} %Long tables

\usepackage{afterpage} %Bruges til at få billeder og figurer til at være samen med kapitlet de hører til
\usepackage{placeins}

% ¤¤ Matematik mm. ¤¤
\usepackage{amsmath,amssymb,stmaryrd} 		% Avancerede matematik-udvidelser
\usepackage{mathtools}						% Andre matematik- og tegnudvidelser
\usepackage{textcomp}                 		% Symbol-udvidelser (f.eks. promille-tegn med \textperthousand )
\usepackage{rsphrase}						% Kemi-pakke til RS-saetninger, f.eks. \rsphrase{R1}
\usepackage[version=3]{mhchem} 				% Kemi-pakke til flot og let notation af formler, f.eks. \ce{Fe2O3}
\usepackage{siunitx}						% Flot og konsistent praesentation af tal og enheder med \si{enhed} og \SI{tal}{enhed}
\sisetup{locale=DE}							% Opsaetning af \SI (DE for komma som decimalseparator) 

% ¤¤ Referencer og kilder ¤¤ %
\usepackage[danish]{varioref}				% Muliggoer bl.a. krydshenvisninger med sidetal (\vref)
%\usepackage{natbib}							% Udvidelse med naturvidenskabelige citationsmodeller
%\usepackage{xr}							% Referencer til eksternt dokument med \externaldocument{<NAVN>}
%\usepackage{glossaries}					% Terminologi- eller symbolliste (se mere i Daleifs Latex-bog)

% ¤¤ Misc. ¤¤ %
\usepackage{lipsum}							% Dummy text \lipsum[..]
\usepackage[shortlabels]{enumitem}			% Muliggoer enkelt konfiguration af lister
\usepackage{pdfpages}						% Goer det muligt at inkludere pdf-dokumenter med kommandoen \includepdf[pages={x-y}]{fil.pdf}

%EMF to PDF conversion
\usepackage{epstopdf}

% Kommentarer og rettelser med \fxnote. Med 'final' i stedet for 'draft' udloeser hver note en error i den faerdige rapport.
\usepackage[footnote,danish,final,nomargin]{fixme}

% Microtype gør at teksten ser pænere ud.
\usepackage{microtype}

\usepackage{listings}
\usepackage{titlesec}
%Usecase and accepttest numbering
%% Se http://en.wikibooks.org/wiki/LaTeX/Counters http://en.wikibooks.org/wiki/LaTeX/Labels_and_Cross-referencing og http://en.wikibooks.org/wiki/LaTeX/Macros for info på hvordan nedenståen er opstået

\newcounter{usecases} %counter
\newcommand{\usecaseset}[1]{\#\refstepcounter{usecases}\arabic{usecases}\label{usecase:#1}} %Tacking an usecase
\newcommand{\usecaseref}[1]{\ref{usecase:#1}} %referencing a usecase

\newcounter{accepttests} %counter
\newcommand{\accepttestset}[1]{\#\refstepcounter{accepttests}\arabic{accepttests}\label{accepttest:#1}}
\newcommand{\accepttestref}[1]{\ref{accepttest:#1}}

%Inkludering af class diagram
\newcommand{\includeclassdiagram}[2]{\includegraphics[width=#1,clip=true,trim=39
40 39 40]{#2}}

%Ens navne på x10 enheder
\newcommand{\xti}{X10}
\newcommand{\xtic}{X10-controller}
\newcommand{\xtie}{X10-enhed}
\newcommand{\xtis}{X10-slave}

\newsubfloat{figure}

%%%% CUSTOM SETTINGS %%%%

\pdfoptionpdfminorversion=6					% Muliggoer inkludering af pdf dokumenter, af version 1.6 og hoejere
\pretolerance=2500 							% Justering af afstand mellem ord (hoejt tal, mindre orddeling og mere luft mellem ord)

% ¤¤ Marginer ¤¤ %
\setlrmarginsandblock{3.5cm}{2.5cm}{*}		% \setlrmarginsandblock{Indbinding}{Kant}{Ratio}
\setulmarginsandblock{2.5cm}{3.0cm}{*}		% \setulmarginsandblock{Top}{Bund}{Ratio}
\checkandfixthelayout 						% Oversaetter vaerdier til brug for andre pakker

%	¤¤ Afsnitsformatering ¤¤ %
\setlength{\parindent}{0mm}           		% Stoerrelse af indryk
\setlength{\parskip}{3mm}          			% Afstand mellem afsnit ved brug af double Enter
\linespread{1,1}							% Linie afstand

% ¤¤ Litteraturlisten ¤¤ %
%\bibpunct[,]{[}{]}{;}{a}{,}{,} 				% Definerer de 6 parametre ved Harvard
% henvisning (bl.a. parantestype og seperatortegn) \bibliographystyle{bibtex/harvard}			% Udseende af litteraturlisten.
\bibliographystyle{plain}			% Udseende af litteraturlisten.

% ¤¤ Indholdsfortegnelse ¤¤ %
\setsecnumdepth{subsection}		 			% Dybden af nummerede overkrifter (part/chapter/section/subsection)
\maxsecnumdepth{subsection}					% Dokumentklassens graense for nummereringsdybde
\settocdepth{subsection} 					% Dybden af indholdsfortegnelsen

% ¤¤ Lister ¤¤ %
\setlist{
  topsep=0pt,								% Vertikal afstand mellem tekst og listen
  itemsep=-1ex,								% Vertikal afstand mellem items
} 

% ¤¤ Visuelle referencer ¤¤ %
\usepackage[colorlinks]{hyperref}			% Danner klikbare referencer (hyperlinks) i dokumentet.
\hypersetup{colorlinks = true,				% Opsaetning af farvede hyperlinks (interne links, citeringer og URL)
    linkcolor = black,
    citecolor = black,
    urlcolor = black
}

% % Sprog opsætning for referencer
\def\figureautorefname{Figur}
\def\tableautorefname{Tabel}

% ¤¤ Opsaetning af figur- og tabeltekst ¤¤ %
\captionnamefont{\small\bfseries\itshape}	% Opsaetning af tekstdelen ('Figur' eller 'Tabel')
\captiontitlefont{\small}					% Opsaetning af nummerering
\captiondelim{. }							% Seperator mellem nummerering og figurtekst
\hangcaption								% Venstrejusterer flere-liniers figurtekst under hinanden
\captionwidth{\linewidth}					% Bredden af figurteksten
\setlength{\belowcaptionskip}{10pt}			% Afstand under figurteksten
		
% ¤¤ Navngivning ¤¤ %
\addto\captionsdanish{
	\renewcommand\appendixname{Appendiks}
	\renewcommand\contentsname{Indholdsfortegnelse}	
	\renewcommand\appendixpagename{Appendiks}
	\renewcommand\appendixtocname{Appendiks}
	\renewcommand\cftchaptername{\chaptername~}				% Skriver "Kapitel" foran kapitlerne i indholdsfortegnelsen
	\renewcommand\cftappendixname{\appendixname~}			% Skriver "Appendiks" foran appendiks i indholdsfortegnelsen
}

% ¤¤ Kapiteludssende ¤¤ %
\definecolor{numbercolor}{gray}{0.7}		% Definerer en farve til brug til kapiteludseende
\newif\ifchapternonum

\makechapterstyle{jenor}{					% Definerer kapiteludseende frem til ...
  \renewcommand\beforechapskip{0pt}
  \renewcommand\printchaptername{}
  \renewcommand\printchapternum{}
  \renewcommand\printchapternonum{\chapternonumtrue}
  \renewcommand\chaptitlefont{\fontfamily{pbk}\fontseries{db}\fontshape{n}\fontsize{25}{35}\selectfont\raggedleft}
  \renewcommand\chapnumfont{\fontfamily{pbk}\fontseries{m}\fontshape{n}\fontsize{1in}{0in}\selectfont\color{numbercolor}}
  \renewcommand\printchaptertitle[1]{%
    \noindent
    \ifchapternonum
    \begin{tabularx}{\textwidth}{X}
    {\let\\\newline\chaptitlefont ##1\par} 
    \end{tabularx}
    \par\vskip-2.5mm\hrule
    \else
    \begin{tabularx}{\textwidth}{Xl}
    {\parbox[b]{\linewidth}{\chaptitlefont ##1}} & \raisebox{-15pt}{\chapnumfont \thechapter}
    \end{tabularx}
    \par\vskip2mm\hrule
    \fi
  }
}											% ... her

\chapterstyle{jenor}						% Valg af kapiteludseende - Google 'memoir chapter styles' for alternativer

% ¤¤ Sidehoved ¤¤ %

\makepagestyle{AAU}							% Definerer sidehoved og sidefod udseende frem til ...
\makepsmarks{AAU}{%
	\createmark{chapter}{left}{shownumber}{}{. \ }
	\createmark{section}{right}{shownumber}{}{. \ }
	\createplainmark{toc}{both}{\contentsname}
	\createplainmark{lof}{both}{\listfigurename}
	\createplainmark{lot}{both}{\listtablename}
	\createplainmark{bib}{both}{\bibname}
	\createplainmark{index}{both}{\indexname}
	\createplainmark{glossary}{both}{\glossaryname}
}
\nouppercaseheads											% Ingen Caps oenskes

\makeevenhead{AAU}{Aarhus Universitet - Gruppe 08}{}{\leftmark}				% Definerer lige siders sidehoved (\makeevenhead{Navn}{Venstre}{Center}{Hoejre})
\makeoddhead{AAU}{\rightmark}{}{Gruppe 08 - Aarhus Universitet}		% Definerer ulige siders sidehoved (\makeoddhead{Navn}{Venstre}{Center}{Hoejre})
\makeevenfoot{AAU}{\thepage}{}{}							% Definerer lige siders sidefod (\makeevenfoot{Navn}{Venstre}{Center}{Hoejre})
\makeoddfoot{AAU}{}{}{\thepage}								% Definerer ulige siders sidefod (\makeoddfoot{Navn}{Venstre}{Center}{Hoejre})
\makeheadrule{AAU}{\textwidth}{0.5pt}						% Tilfoejer en streg under sidehovedets indhold
\makefootrule{AAU}{\textwidth}{0.5pt}{1mm}					% Tilfoejer en streg under sidefodens indhold

\copypagestyle{AAUchap}{AAU}								% Sidehoved for kapitelsider defineres som standardsider, men med blank sidehoved
\makeoddhead{AAUchap}{}{}{}
\makeevenhead{AAUchap}{}{}{}
\makeheadrule{AAUchap}{\textwidth}{0pt}
\aliaspagestyle{chapter}{AAUchap}							% Den ny style vaelges til at gaelde for chapters
															% ... her
															
\pagestyle{AAU}												% Valg af sidehoved og sidefod


%%%% CUSTOM COMMANDS %%%%

% ¤¤ Billede hack ¤¤ %
\newcommand{\figur}[4]{
		\begin{figure}[H] \centering
			\includegraphics[width=#1\textwidth]{billeder/#2}
			\caption{#3}\label{#4}
		\end{figure} 
}

% ¤¤ Specielle tegn ¤¤ %
\newcommand{\grader}{^{\circ}\text{C}}
\newcommand{\gr}{^{\circ}}
\newcommand{\g}{\cdot}


%%%% ORDDELING %%%%

\hyphenation{}

% % listings setup
\lstset{
tabsize=4,
frame=single,
columns=fullflexible,
linewidth=\textwidth
}

% % % sørg for at floats er på siden inden en ny section starter
%\newcommand{\sectionbreak}{\clearpage}
%\newcommand{\subsectionbreak}{\FloatBarrier}

%Code handling - Colors lstlisting

\lstset{language=C++,
                basicstyle=\ttfamily,
                keywordstyle=\color{blue}\ttfamily,
                stringstyle=\color{red}\ttfamily,
                commentstyle=\color{green}\ttfamily,
                morecomment=[l][\color{magenta}]{\#}
}


\title{Samarbejdskontrakt for semesterprojekt}
\author{E3PRJ3 Gruppe 08}
\date{\today{} \currenttime{}}

\begin{document}

\underline{ \huge \bfseries Samarbejdskontrakt for PRJ3-gruppe 08 \\[0.4cm] }

Jeg erklærer herved, ved min herunder satte underskrift eller ved mit overfor gruppen mundtligt fremsatte løfte, at overholde de af gruppe 08 enstemmigt vedtagede forskrifter for samarbejde og mødepligt.

\begin{enumerate}[label=\bfseries § \arabic*.]

	\item Alle gruppemedlemmer er pålagt mødepligt, hvor afbud ellers skal meldes på forhånd, så der ikke ventes unødvendigt på nogen.
		\begin{enumerate} [label=\bfseries stk. \arabic*:]
		\setcounter{enumii}{1}
		\item Ved udeblivelse fra møde eller planlagt arbejde uden forudgående besked, afregnes der ved gældende bødetarif.
		\item Bødetariffen er ved møde fastsat til en spiselig snack af mindre størrelse.
		\item Bødestraffen kan frafalde i tilfælde af force majeure.
		\item Der kan meldes afbud igennem facebookgruppen, campusnet, skriftligt eller mundtligt. Afbud skal ske senest ved mødet/gruppearbejdets begyndelse.
		\item Opleves et stort fravær for et gruppemedlem, skal fraværet uddybes af medlemmet. Mistanke om misbrug af §1, stk. 5 medfører også ubetinget uddybning.
		\item Ved gentagne overtrædelser af §1, stk. 1, §1, stk. 6, §2, stk. 1 og §3, stk. 2 kan gruppemedlemmer tvinges ud af gruppen, i samråd med vejleder, hvis det vedtages enstemmigt af de resterende gruppemedlemmer.
		\end{enumerate}
	
	\item Det påhviler alle gruppemedlemmer at deltage aktivt i møderne eller hvad der ellers findes af planlagt arbejde, så praktisk som det er muligt.
		\begin{enumerate} [label=\bfseries stk. \arabic*.]
		\setcounter{enumii}{1}
		\item Det forventes at alle medlemmer møder med positiv indstilling og at arbejdsmiljøet holdes behageligt.
		\item Det påhviler alle medlemmer at føre kortfattet dagbog over gennemført arbejde, hvor hvert møde/planlagt arbejde. 
		\end{enumerate}
		
	\item Det påhviler alle gruppemedlemmer at overholde de aftaler der bliver indgået internt i gruppen, herunder fordeling af arbejde eller opgaver.
		\begin{enumerate} [label=\bfseries stk. \arabic*.]
		\setcounter{enumii}{1}
		\item Manglende overholdelse af aftaler, straffes med bøde jf. §1, stk. 2 og 4.
		\item Et gruppemedlems valg af linje, være det sig elektro, IKT eller stærkstrøm, medfører ikke eneret på opgaver på disse felter. Dog bør medlemmernes ønsker om opgavefordelingen tages til efterretning ved fordelingen.
		\item Skulle der opstå problemer ved løsning af tildelt opgave, skal dette meldes hurtigt muligt, melding sker ligeledes §1, stk. 5.
		\end{enumerate}	
	
	\item Det påhviler alle gruppemedlemmer at møde forberedt op til øvelserne.
		%\begin{enumerate}[label=\bfseries stk. \arabic*.]
		\setcounter{enumii}{1}
		%\item 
		%\end{enumerate}
	
	\item Formand vælges internt i gruppen, ved demokratisk valg blandt de opstillede.
		\begin{enumerate}[label=\bfseries stk. \arabic*.]
		\setcounter{enumii}{1}
		\item Formanden har det overordnede overblik og ansvar for at gruppen holdes på sporet.
		\item Formanden har desuden ansvaret for at der indkaldes til møder/arbejdsmøder og møder med gruppens vejleder. Desuden også ansvar for lokalebooking.
		\item I situationer hvor en problemstilling ikke kan afgøres ved afstemning, har formanden det sidste ord. Hvis en diskussion tager mere end 30 minutter, kan formanden vælge at holde afstemning.
		\item Resten af gruppen kan enstemmigt vedtage et mistillidsvotum imod formanden, hvorefter nyt formandsvalg afholdes jf. §5, stk. 1.
		\item Referent/sekretær vælges af formanden i samråd med gruppen.
		\end{enumerate}

	\item Før et gruppemødes afholdelse, skal agenda foreligge gruppemedlemmerne til godkendelse/gennemlæsning.
		\begin{enumerate}[label=\bfseries stk. \arabic*.]
		\setcounter{enumii}{1}
		\item Ved mødets påbegyndelse godkendes referatet fra foregående møde.
		\item Ved hvert mødes afslutning skal referat udformes af referenten, jf. §5, stk. 6. Endvidere skal nyt møde aftales. 
		\end{enumerate}

			
\end{enumerate}

\chapter*{Deltageres underskrift}

\phantom{Luft}

\phantom{Luft}

\begin{table}[H]
	\centering
		\begin{tabular}{c c c}
		&&\\
		&&\\
		&&\\
		&&\\
			\underline{\phantom{mmmmmmmmmmmmmm}} && \underline{\phantom{mmmmmmmmmmmmmm}} \\
			Andreas Laursen && Dannie Lehmann\\
			20104209 && 201270024\\
			&&\\
			&&\\
			&&\\
			&&\\
			&&\\
			\underline{\phantom{mmmmmmmmmmmmmm}} && \underline{\phantom{mmmmmmmmmmmmmm}} \\
			Jens Rix Jørgensen && Mathias Jessen\\
			10253 && 201270725\\
			&&\\
			&&\\
			&&\\
			&&\\
			&&\\
			\underline{\phantom{mmmmmmmmmmmmmm}} && \underline{\phantom{mmmmmmmmmmmmmm}} \\
			Morten Møller Christensen && Simon Mouridsen\\
			20062548 && 201303580\\
			&&\\
			&&\\
			&&\\
			&&\\
			&&\\
		 							& \underline{\phantom{mmmmmmmmmmmmmm}} 	&			\\														
									& Stine Skaarup Høgsberg\\
									& 201270943								
		\end{tabular}
\end{table}

\end{document}

